\documentclass{article}

\usepackage[utf8]{inputenc}
\usepackage[german]{babel}
\usepackage{fancyhdr}
\usepackage{lastpage}

\author{Florian Bär}
\title{Fallstudie Minicase}
\subtitle{Information Security Fundamentals}
\pagenumbering{arabic}

\newcommand*{\thead}[1]{\multicolumn{1}{c}{\bfseries #1}}

\rfoot{Page \thepage \hspace{1pt} of \pageref{LastPage}}

\begin{document}
\tableofcontents
	
\maketitle

\section{Einführung zum Minicase}

Die Familie Meier wohnt in einem dreistöckigen Haus im zweiten Stockwerk. Insgesamt teilen 6 Wohnungen das Haus auf. Eine fortschrittliche Familie aus dem Mittelstand mit bester Reputation. Für diese Familie sollte einen Massnahmenplan für allfällige Risiken gemäss der Aufgabenstellung "Fallstudie: Die Heim PC Lösung" erstellt werden. Für die Eintrittswahrscheinlichkeit und der potenzielle Schaden (Schadensausmass) der Ereignisse wird eine Skala von 1 (sehr kleine(r) Wahrscheinlichkeit/Schaden) bis 4 (sehr grosse(r) Wahrscheinlichkeit/Schaden) verwendet.

\section{Fallstudie ISF}

\subsection{Risiken}

\subsubsection{Beschreibung der Risiken}



Es folgt eine Liste mit Dingen, welche schiefgehen können:
1. Das Wireless brauch WEP. WEP ist veraltet und nicht sicher. Es ist ein upgrade
zu WPA2 empfohlen.
•intritnumswahrscheinlichkeit: 4 - Da WEP schon veraltet ist und in wenigen
Sekunden mit einem mittelklassigen Computer geknackt werden kann ist die
Eintrittswahrscheinlichkeit hoch.
• Schaden: 4 - Mit Zugriff auf das Netzwerk ist dem “Hacker” alles Möglich
(Zugriff auf Computer, Kameras etc.).
• RISIKO - Extrem (16)
2. Die Spiele und Tools, welche Onkel Özutück mitbringt, könnten gecrackt sein.
Cracks beinhalten ein Sicherheitsrisiko. Selbst wenn Sie nicht gefährlich sind, dann
sind diese illegal verwendet.
• Eintrittswahrscheinlichkeit: 4 - Sehr hoch, da im Text steht, dass der Onkel
Özukück gratis Software mitbringt.
• Schaden: 4 - Da Cracks häufig infiziert sind, ist das Risiko für das infizieren
mit einer Schadsoftware sehr gross.
• RISIKO - Extrem (16)
3. Da Jan ein Computerfreak ist, versucht er bestimmt immer neue Dinge. Dabei
wird das Sicherheitsrisiko von diesen “Nerd”-Tools oftmals unterschätzt.
• Eintrittswahrscheinlichkeit: 3 - Es gibt auch auf Github und anderen Clouddi-
ensten viele Schadsoftware, welche als kleine Tools getarnt sind.
• Schaden: 4 - Wenn der PC mit Schadsoftware infiziert wurde, dann ist dieser
Computer dieser ausgeliefert.
• RISIKO - Hoch (12)
4. Dora ist mit 12 Jahren ein junges Kind. Vor allem bei jungen Kindern sollte der
Internetkonsum kontrolliert werden. Dazu gibt es passende Kinderschutzsoftware.
• Eintrittswahrscheinlichkeit: 4 - Da Kinder tendenziell sich am PC explorativ
verhalten, kann Dora einfach auf zwilichtige Webseiten gelangen.
• Schaden: 4 - Unter Umständen
• RISIKO - Extrem (16)
5. Die Computer müssten auf einem NAS oder einem sonstigen Server ein Backup
haben. Falls einmal eine HD kaputt gehen sollte, müssen die Daten wiederherstellbar
sein.
• Eintrittswahrscheinlichkeit: 3
• Schaden: 3
• RISIKO - Hoch (9)
6. Die Kameras könnten von einem Billighersteller sein, welcher die Sicherheitsysteme
des Systems nicht sehr verantwortungsvoll implementiert. Diese könnten dann
“gehackt” werden und einen grossen Eingriff in die Privatsphäre der Familie.
Eintrittswahrscheinlichkeit: 2
• Schaden: 3
• RISIKO - Mittel (6)
7. Der Desktop PC der Eltern sollte nicht gleichzeitig als PC und als Printserver
verwendet werden. Server öffnen normalerweise Ports und Services für das Netzwerk
und können so ein Sicherheitsrisiko darstellen.
• Eintrittswahrscheinlichkeit: 2 - Games und Server öffnen Services für die
Multiplayerfähigkeit. Diese stellen potenziell ein Sicherheitsrisiko dar.
• Schaden: 3 -
• RISIKO - Mittel (6)
8. Windows Betriebssysteme besitzen zwar den Bitlocker, jedoch ist dieser Standard-
mässig nicht eingeschaltet und kann auch nur von forgeschrittenen Usern bedient
werden. Dadurch sind die Daten auf den Disks nicht verschlüsselt und können unter
Umständen ausgelesen werden.
• Eintrittswahrscheinlichkeit: 2
• Schaden: 3
• RISIKO - Hoch (6)
9. Bei einem Brand wären alle Computer zerstört und es könnten keine Daten wieder-
hergestellt werden.
• Eintrittswahrscheinlichkeit: 2
• Schaden: 4
• RISIKO - Mittel (8)
10. Benutzen die Kinder und die Eltern die selbe Mail? Dies könnte allenfalls auch zu
einem Risiko werden. Nicht alle Mails sind für die Augen der Kinder gedacht.
• Eintrittswahrscheinlichkeit: 3
• Schaden: 2
• RISIKO - Mittel (6)


\subsubsection{Risikomatrix}

\begin{tabular}{|r|c|c|c|}
	\hline
	\thead{Risiko ID} & \thead{EW\footnote{Eintrittswahrscheinlichkeit}} & \thead{SG\footnote{Schadensgrösse}} & \thead{R\footnote{Risiko}} \\ \hline
	1 &  \cellcolor{blue!25}b & c & d \\ \hline
	2 & b & c & d \\ \hline
	3 &  \cellcolor{blue!25}b & c & d \\ \hline
	4 & b & c & d \\ \hline
	5 &  \cellcolor{blue!25}b & c & d \\ \hline
	6 & b & c & d \\ \hline
	7 &  \cellcolor{blue!25}b & c & d \\ \hline
	8 & b & c & d \\ \hline
	9 &  \cellcolor{blue!25}b & c & d \\ \hline
	10 & b & c & d \\ \hline
\end{tabular}




\subsection{Massnahmen}

\subsubsection{Beschreibung der Massnahmen}

\subsubsection{Risikomatrix nach Massnahmen}



\section{Fazit}



\end{document}
