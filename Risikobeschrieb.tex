\section{Fallstudie ISF}

\subsection{Risiken}

\subsubsection{Beschreibung der Risiken}

Es folgt eine Liste mit Dingen, welche Sicherheitsrisiken beinhalten:
\begin{enumerate}
\item Das Wireless brauch WEP. WEP ist veraltet und nicht sicher. Es ist ein upgrade
zu WPA2 empfohlen.
\item Die Spiele und Tools, welche Onkel Özutück mitbringt, könnten gecrackt sein.
Cracks beinhalten ein Sicherheitsrisiko. Selbst wenn Sie nicht gefährlich sind, dann
sind diese illegal verwendet.
\item Da Jan ein Computerfreak ist, versucht er bestimmt immer neue Dinge. Dabei wird das Sicherheitsrisiko von diesen “Nerd”-Tools oftmals unterschätzt.
\item Dora ist mit 12 Jahren ein junges Kind. Vor allem bei jungen Kindern sollte der
Internetkonsum kontrolliert werden. Dazu gibt es passende Kinderschutzsoftware.
\item Die Computer müssten auf einem NAS oder einem sonstigen Server ein Backup
haben. Falls einmal eine HD kaputt gehen sollte, sind die Daten nicht wiederherstellbar.
\item Die Kameras könnten von einem Billighersteller sein, welcher die Sicherheitsysteme
des Systems nicht sehr verantwortungsvoll implementiert. Diese könnten dann
“gehackt” werden und einen grossen Eingriff in die Privatsphäre der Familie.
\item Der Desktop PC der Eltern sollte nicht gleichzeitig als PC und als Printserver
verwendet werden. Server öffnen normalerweise Ports und Services für das Netzwerk
und können so ein Sicherheitsrisiko darstellen.
\item Windows Betriebssysteme besitzen zwar den Bitlocker, jedoch ist dieser Standard-
mässig nicht eingeschaltet und kann auch nur von forgeschrittenen Usern bedient
werden. Dadurch sind die Daten auf den Disks nicht verschlüsselt und können unter
Umständen ausgelesen werden.
\item Bei einem Brand wären alle Computer zerstört und es könnten keine Daten wieder-
hergestellt werden.
\item Benutzen die Kinder und die Eltern die selbe Mail? Dies könnte allenfalls auch zu
einem Risiko werden. Nicht alle Mails sind für die Augen der Kinder gedacht.
\end{enumerate}
\newpage

\begin{landscape}

\subsubsection{Risikomatrix}
\begin{tabularx}{\columnwidth}{|r|c|X|c|X|c|}
	\hline
	\textbf{ID} & \textbf{EW\footnote{Eintrittswahrscheinlichkeit}} & \textbf{Begründung} & \textbf{SG\footnote{Schadensgrösse}} & \textbf{Begründung} & \textbf{Risiko} \\ 
	\hline		
	1 &  \cellcolor{red}4 & Da WEP schon veraltet ist und in wenigen Sekunden mit einem mittelklassigen Computer geknackt werden kann ist die Eintrittswahrscheinlichkeit hoch. & \cellcolor{red}4 & Mit Zugriff auf das Netzwerk ist dem “Hacker” alles Möglich	(Zugriff auf Computer, Kameras etc.).  & \cellcolor{red}16 \\ \hline
	2 &  \cellcolor{red}4 & Sehr hoch, da im Text steht, dass der Onkel Özukück gratis Software mitbringt.  & \cellcolor{red}4 & Da Cracks häufig infiziert sind, ist das Risiko für das infizieren mit einer Schadsoftware sehr gross.  & \cellcolor{red}16 \\ \hline
	3 &  \cellcolor{orange}3 &  Es gibt auch auf Github und anderen Clouddiensten viele Schadsoftware, welche als kleine Tools getarnt sind.  & \cellcolor{red}4 & Wenn der PC mit Schadsoftware infiziert wurde, dann ist dieser Computer dieser ausgeliefert.  & \cellcolor{red}12 \\ \hline
	4 &  \cellcolor{red}4 & Da Kinder tendenziell sich am PC explorativ verhalten, kann Dora einfach auf zwilichtige Webseiten gelangen.  & \cellcolor{red}4 & Der Besuch von "schlechten" Webseiten kann die Zukunft von Dora massiv beeinflussen und im schlimmsten Fall dazu führen, dass sie einmal im Gefängnis landet. & \cellcolor{red}16 \\ \hline
	5 &  \cellcolor{orange}3 & Es kann immer einmal ein Laptop auf den Boden fallen. Falls dann eine HD kaputt geht, sind allenfalls gespeicherte Fotos auf diesem Computer verloren. Auch von einer allfälligen Fehlproduktion ist man nicht geschützt. & \cellcolor{orange}3 & Der materielle Schaden ist bei der Familie nicht sehr gross. Allerdings kann der Verlust von Familienfotos sehr schmerzhaft sein. & \cellcolor{orange}9 \\ \hline
	6 &  \cellcolor{yellow}2 & Auch Billigkamerahersteller bemüht, das Sicherheit einem Minimalstandard entspricht. & \cellcolor{orange}3 & Das Interesse von einem Hacker, sich Zugriff auf eine Kamera zu verschaffen ist allerhöchstens von perversem Interesse. & \cellcolor{yellow}6 \\ \hline\textsl{}
	7 &  \cellcolor{yellow}2 & Games und Server öffnen Services für die Multiplayerfähigkeit. Diese stellen potenziell ein Sicherheitsrisiko dar. & \cellcolor{red}4 & Wenn man mit einem Bufferoverflow oder anderen Code-Injection sich Zugriff auf einen PC erschleichen kann, dann ist man Herr über dieses Gerät. & \cellcolor{orange}8 \\ \hline
	8 &  \cellcolor{yellow}2 & Damit man an die Daten des Computers kommt, muss man zuerst Zugriff auf diesen Computer haben. Dies ist nur bei einem Einbruch oder einem Diebstahl möglich. & \cellcolor{orange}3 & Wenn man allerdings einmal Zugriff auf diesen Computer hat, dann ist es einer Fremdperson möglich, alle Daten\footnote{inkl. Passwörter und persönliche Unterlagen} auszulesen.  & \cellcolor{yellow}6 \\ \hline
	9 &  \cellcolor{yellow}2 & Dass alle Daten und Backups zerstört werden ist grundsätzlich nur bei einem Brand möglich. Diese sind in der Schweiz zum Glück selten. & \cellcolor{red}4 & Wenn ein Brand ausbrechen würde, dann wären alle Daten und auch deren (allenfalls gemachten) Back-Ups zerstört. & \cellcolor{orange}8 \\ \hline
	10 &  \cellcolor{orange}3 & Normalerweise interessieren sich die Kinder nicht für die Mails der Eltern. Und um Schaden zu verursachen, müsste dies Mutwillig seitens der Kinder geschehen. & \cellcolor{yellow}2 & Die Eltern haben normalerweise nicht viele (bis keine) Information, welche Schaden anrichten können. & \cellcolor{yellow}6 \\ \hline
\end{tabularx}

\newpage

\end{landscape}